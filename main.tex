\documentclass[11pt, english]{article}
\usepackage{graphicx}
\usepackage[colorlinks=true, linkcolor=blue]{hyperref}
\usepackage[english]{babel}
\selectlanguage{english}
\usepackage[utf8]{inputenc}
\usepackage[svgnames]{xcolor}



\usepackage{listings}
\usepackage{afterpage}
\pagestyle{plain}

\definecolor{dkgreen}{rgb}{0,0.6,0}
\definecolor{gray}{rgb}{0.5,0.5,0.5}
\definecolor{mauve}{rgb}{0.58,0,0.82}

%\lstset{language=R,
%    basicstyle=\small\ttfamily,
%   stringstyle=\color{DarkGreen},
%    otherkeywords={0,1,2,3,4,5,6,7,8,9},
%    morekeywords={TRUE,FALSE},
%    deletekeywords={data,frame,length,as,character},
%    keywordstyle=\color{blue},
%    commentstyle=\color{DarkGreen},
%}

\lstset{frame=tb,
language=R,
aboveskip=3mm,
belowskip=3mm,
showstringspaces=false,
columns=flexible,
numbers=none,
keywordstyle=\color{blue},
numberstyle=\tiny\color{gray},
commentstyle=\color{dkgreen},
stringstyle=\color{mauve},
breaklines=true,
breakatwhitespace=true,
tabsize=3
}

\usepackage{here}


\textheight=21cm
\textwidth=17cm
%\topmargin=-1cm
\oddsidemargin=0cm
\parindent=0mm
\pagestyle{plain}

%%%%%%%%%%%%%%%%%%%%%%%%%%
% La siguiente instrucción pone el curso automáticamente%
%%%%%%%%%%%%%%%%%%%%%%%%%%

\usepackage{color}
\usepackage{ragged2e}

\global\let\date\relax
\newcounter{unomenos}
\setcounter{unomenos}{\number\year}
\addtocounter{unomenos}{-1}
\stepcounter{unomenos}
\gdef\@date{ Course \arabic{unomenos}/ 2019}

\begin{document}

\begin{titlepage}

\begin{center}
\vspace*{-1in}
\begin{figure}[htb]
\begin{center}
\includegraphics[width=8cm]{logo}
\end{center}
\end{figure}

INFORMATION \& NETWORK DYNAMICS (INDY) LAB - \@date\\
\vspace*{0.4in}
\begin{large}
MATRIX FACTORISATION\\
\end{large}
\vspace*{0.2in}
\begin{Large}
\textbf{Optimal Regularizer} \\
\end{Large}
\vspace*{0.3in}
\begin{large}
Ritabrata Ray \\
\end{large}
\vspace*{0.3in}
\rule{80mm}{0.1mm}\\
\vspace*{0.1in}
\begin{large}
IIT Kharapur \\
\end{large}
\includegraphics[width=2cm]{LogoFac.jpg}
\end{center}
\end{titlepage}

\newcommand{\CC}{C\nolinebreak\hspace{-.05em}\raisebox{.4ex}{\tiny\bf +}\nolinebreak\hspace{-.10em}\raisebox{.4ex}{\tiny\bf +}}
\def\CC{{C\nolinebreak[4]\hspace{-.05em}\raisebox{.4ex}{\tiny\bf ++}}}

\tableofcontents
\newpage
\section{Objective}
In this work, we aim to find the optimal values of the regularization strength of different parameters with different frequencies for the famous Matrix Completion Problem for the Netflix Problem. The parameters here are the regularization strengths of each embedding vector for each user and each movie.  
%Our Optimisation problem is as follows : \\
%\begin{center}
%\min\limits_{x \in R^d} f(x) & = \[ \frac{1}{n} %\sum\limits_{i=1}^{n} f \limits_{i}(x) \]

%\end{center} 
%\\ Convex function definition & L-lipchitz smooth  
\section{Matrix Factorization for the Netflix Problem}
w our limitations.


\subsection{GRASP and its parameter $\alpha$}
The values of the objective function has been taken in the following table:


\subsubsection{Graphic representation and numeric data}

We are going to see in the next code in R (the graphic representation will be done by a box plot). \\


\subsubsection{Hypothesis contrasting}
Now we will have done the hypothesis contrasting, which our null hypothesis $H_0$ is 
\subsection{Tabu Search (\textit{First}) and its parameter \textit{tenure}.}
We have studied forty \textit{tenure} (from 1 to 40) for \subsubsection{Graphic and numeric representation}
We are going to see it in the next script of R (the graphic representation will be done by a box plot).


\subsubsection{Hypothesis contrasting.}
Now we will have done the hypothesis contrasting, which our null hypothesis $H_0$ is 
\subsection{Tabu Search (\textit{Best}) and its parameter \textit{tenure}.}
We have studied forty \textit{tenure} (from 1 to 40) the 


\subsubsection{Graphic representation and numeric data}
We are going to see it in the next script of R (the graphic representation will be done by a box plot).


\section{Comparative between GRASP and TABU.}
In the next section we are going to discus our target from the beggining, see what is the best algorithms, GRAPS or 

\subsubsection{Graphic and numeric representation}

We are going to see it in the next script of R (the graphic 


\subsubsection{Hypothesis contrasting.}
Now, we

\section{Conclusions and conjecture.}
Our algorithm GRASP,which we have done with our teacher Rafael Martí, has resulted to be the winner. One possibility is that we have a good implementation of the construction of the solution and a good implementation of the local search. \\


%uoooooooooooooooo tumadreuooooooooooooooooooo UOOOOOOOOOOOOOOOOOOOOOOOOOOOOOOOOOOOOOOOOO
%AL FIN SE TERMINA ESTA PUTA MIERDA!!!!
%USEGREAS OSTOJEOGIRN ojeogiek


\end{document}
